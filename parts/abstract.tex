As display technologies evolve and high-resolution screens become more available, the desirability of images and videos with high perceptual quality grows in order to properly utilize such advances. At the same time, the market for illustrated mediums, such animations and comics, has been in steady growth over the past years. Based on these observations, we were motivated to explore the super-resolution task in the niche of drawings. In absence of original high-resolution imagery, it is necessary to use approximate methods, such as interpolation algorithms, to enhance low-resolution media. Such methods, however, can produce undesirable artifacts in the reconstructed images, such as blurring and edge distortions. Recent works have successfully applied deep learning to this task, but such efforts are often aimed at real-world images and do not take in account the specifics of illustrations, which emphasize lines and employ simplified patterns rather than complex textures, which in turn makes visual artifacts introduced by algorithms easier to spot. With these differences in mind, we evaluated the effects of the choice of loss functions in order to obtain accurate and perceptually pleasing results in the super-resolution task for comics, cartoons, and other illustrations. Experimental evaluations have shown that a loss function based on edge detection performs best in this context among the evaluated functions, though still showing room for further improvements.

% As display technologies evolve and high-resolution screens become more available, the desirability of images and videos with high perceptual quality grows in order to properly utilize such advances. \todo{Falar do aumento de conteudo de animacoes, e da grande atencao que esta sendo dada aos quadrinhos e manga. Deixar claro aqui que o trabalho eh sobre animacoes} In absence of original high-resolution imagery, it is necessary to use approximate methods, such as interpolation algorithms, to enhance low-resolution media. \todo{Qual o problema em resolver usando metodos aproximados?} While recent works have successfully applied deep learning to this task, such efforts are often aimed at real-world pictures and do not consider the specifics of illustrations, which emphasize lines and employ simplified patterns rather than complex textures, thereby emphasizing imperfections originated from the process \todo{não entendi essa parte após a vírgula}. With these differences in mind, we evaluated the effects of the choice of loss functions in order to obtain accurate and perceptually pleasing results in the task of \review{super-resolving - esta correto?} comics, cartoons, and other illustrations. Experimental evaluations have shown that a loss function based on edge detection performs best in this context among the evaluated functions, though still showing room for further improvements.